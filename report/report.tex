\documentclass[11pt]{article}
\usepackage[margin=1.25in]{geometry}
\usepackage{graphicx,float}
\usepackage{listings}

%opening
\title{Tree Drawing}
\author{Vincent La}

\begin{document}

\maketitle

\section*{Binary Tree Drawing}
Trees with at most two child nodes are used widely in computer science. The simplicity of their structure easily lends to mathematical analysis about algorithms operating on binary trees. Given the vast utility of this tree, many have tried to define algorithms which draw binary trees.

\begin{figure}[H]
    \centering
    \def\svgwidth{\columnwidth}
    \input{tree_2.pdf_tex}
\end{figure}

\section*{N-Ary Tree Drawing}
\section*{Code}
\lstinputlisting[language=C++]{tree.cpp}

\end{document}
